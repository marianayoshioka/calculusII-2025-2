% Pacotes úteis
\usepackage[utf8]{inputenc}
\usepackage[T1]{fontenc}
\usepackage{amsmath, amssymb, amsthm}
\usepackage[most]{tcolorbox}
\usepackage{physics}        % derivadas, integrais mais bonitinhas
\usepackage{enumitem}       % personalizar listas
\usepackage{graphicx}       % imagens
\usepackage{caption}		% legendas de imagens
\usepackage{hyperref}       % links no sumário
\usepackage{geometry}       % margens
\usepackage{mathtools}
\usepackage{listings}
\lstset{numbers=left, captionpos=b}
\usepackage{xcolor}

\geometry{margin=2.5cm}

% Ambientes 
\renewcommand{\contentsname}{Sumário}
\renewcommand{\lstlistingname}{Código}

\newtheoremstyle{meuTeorema}  % nome
{1em}    % Espaço acima
{1em}    % Espaço abaixo
{\itshape} % Fonte do corpo
{}       % Recuo
{\bfseries} % Fonte do cabeçalho
{.}      % Pontuação após cabeçalho
{.5em}   % Espaço após cabeçalho
{}       % Cabeçalho especifico

\theoremstyle{meuTeorema}
\newtheorem{teorema}{Teorema}[section]

\theoremstyle{definition}
\newtheorem{definicao}{Definição}[section]

\theoremstyle{remark}
\newtheorem{exemplo}{Exemplo}[section]

% Ambiente para exercício (versão simples)
\newenvironment{exercicio}[1]{
	\vspace{0.5cm}
	\noindent\textbf{Exercício #1. }\itshape}{\vspace{0.3cm}}

\newtcolorbox{solucao}[1][]{
	enhanced,
	breakable,
	colback=brown!5!white,  
	colframe=brown!90!black,
	fonttitle=\bfseries,
	title=Solução,
	boxrule=0.4pt,
	arc=0pt,
	outer arc=0pt,
	coltitle=black
}

%Ambiente para códigos
\lstdefinestyle{python_estiloso}{
	language=Python,
	basicstyle=\ttfamily\small,   % fonte monoespaçada menor
	keywordstyle=\color{blue},
	commentstyle=\color{green!50!black},
	stringstyle=\color{red},
	backgroundcolor=\color{brown!5!white}, % mesmo tom do tcolorbox
	numbers=left,
	numberstyle=\tiny\color{gray},
	stepnumber=1,
	numbersep=5pt,
	frame=single,
	rulecolor=\color{brown!90!black},
	breaklines=true,
	tabsize=4,
	showstringspaces=false
}