\documentclass[../main.tex]{subfiles}
\begin{document}
	
	\section{Lista 6}
	\begin{exercicio}{1}
		(T. Apostol; seção 10.5; exercícios 1, 4 e 9)
		
		Desenhar em ambiente computacional o campo e o caminho. Resolver manualmente e computacionalmente a integral de linha. Escolha 2 (ou mais) pontos da curva para indicar ambos os vetores do integrando: tangente da curva e o campo no ponto específico.
		
		\begin{enumerate}
			\item[1.] $f(x,y)=(x^2-2xy)\textbf{i} + (y^2-2xy)\textbf{j}$, de $(-1,1)$ até $(1,1)$ ao longo da parábola $y=x^2$.
			\item[4.] $f(x,y)=(x^2+y^2)\textbf{i}+(x^2-y^2)\textbf{j}$, de $(0,0)$ até $(2,0)$ ao longo da curva $y=1-|1-x|$.
			\item[9.] $\int_C (x^2-2xy)dx+(y^2-2xy)dy$, onde $C$ é o caminho de $(-2,4)$ até $(1,1)$ ao longo da parábola $y=x^2$.   
		\end{enumerate}
	\end{exercicio}
	
	\begin{exercicio}{2}
		(T. Apostol; seção 10.9; exercícios 2 e 8)
		
		\begin{enumerate}
			\item[2.] Encontre a quantidade de trabalho feito pela força $f(x,y)=(x^2-y^2)\textbf{i}+2xy\textbf{j}$ ao mover a partícula (no sentido anti-horário) uma vez ao redor do quadrado limitado pelos eixos de coordenadas e pelas linhas $x=a$ e $y=a$, $a>0$.
			\item[8.] Calcule a integral com respeito ao comprimento de arco $\int_C y^2 ds$, onde $C$ tem a equação vetorial
			\[
			\alpha(t)=a(t-\sin(t))\textbf{i} + a(1-\cos(t))\textbf{j}, 0\leq t\leq 2\pi
			\]
		\end{enumerate}
	\end{exercicio}
	
	\begin{exercicio}{3}
		(T. Apostol; seção 10.13; exercício 6)
		\begin{enumerate}
			\item[6.] Um campo de força $f$ é definido no $\mathbb{R}^3$ pela equação
			\[
			f(x,y,z)=y\textbf{i} + z\textbf{j} +yz\textbf{k}
			\]
			\begin{enumerate}[label=\alph*)]
				\item Determine se $f$ é conservativa ou não.
				\item Calcule o trabalho feito ao mover a partícula pela curva descrita por
				\[
				\alpha(t)=\cos(t)\textbf{i}+\sin(t)\textbf{j}+e^t\textbf{k}
				\]
				Quando $t$ varia de $0$ até $\pi$.
			\end{enumerate}
		\end{enumerate}
	\end{exercicio}
	
	\begin{exercicio}{4}
		(T. Apostol; seção 10.18; exercícios 3 e 9)
		
		Determine se $f$ é gradiente de um campo escalar ou não. Quando $f$ é gradiente, encontre uma função potencial correspondente $\varphi$.
		\begin{enumerate} 
			\item[3.] $f(x,y)=(2xe^y+y)\textbf{i}+(x^2e^y+x=2y)\textbf{j}$.
			\item[9.] $f(x,y,z)=3y^4z^2\textbf{i}+4x^3z^2\textbf{j}-3x^2y^2\textbf{k}$
		\end{enumerate}
	\end{exercicio}
\end{document}